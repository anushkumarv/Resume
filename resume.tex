\documentclass[11pt,a4paper]{article}
\usepackage[T1]{fontenc}
% \usepackage[latin1]{inputenc}
\usepackage{charter}
\usepackage[english]{babel}
\usepackage[centering,includefoot,margin=0.75cm]{geometry}
\usepackage{xcolor}
\usepackage{calc,blindtext}
\usepackage{setspace}
\setlength{\parindent}{0pt}
\usepackage{hyperref}

\begin{document}
\begin{center}
    \textbf{\huge ANUSH KUMAR VENKATESH}
    \\ Email : anushkumarv3 [at] gmail [dot] com
\end{center}
\colorbox{gray}{\makebox[\textwidth-2\fboxsep][l]{\bfseries\color{white} Education}}
\vspace{-20pt}
  \begin{itemize}
      \item \textbf{University Of Colorado Boulder} \emph{, Boulder, USA}
      \\ Master of Science in Computer Science - \emph{incoming student for fall 2022}
  \end{itemize}
  \begin{itemize}
    \item \textbf{Ramaiah Institute Of Technology}\emph{, Bengaluru, India} \hfill \textit{Aug 2015 - Jul 2019}
    \\ B.E in Information Science and Engineering \emph{CGPA : 9.48}
  \end{itemize}
{
\colorbox{gray}{\makebox[\textwidth-2\fboxsep][l]{\bfseries\color{white} Work Experience}}
  \begin{itemize}
    %   \setlength\itemsep{0em}
      \item \textbf{J.P.Morgan \& Chase} | \emph{Associate Software Engineer} \hfill \textit{Jan 2019 - July 2022}  
      \begin{itemize}
          \item Contributed to the development of SACCR (heuristic-based) and IMM (internal market model based) calculators which helps in calculating the \emph{counterparty credit risk} of multiple clients the firm trades with.
          \item Primarily worked on refactoring the legacy code responsible for the distribution and parallel processing of multiple clients same time. It helped in reducing the execution time by 4 hours.
        %   \item Contributed to a new heuristic-based calculator. It takes into account market parameters and determines the risk associated with multiple clients the firm actively trades with.
        %   \item Refactored the legacy code responsible for the distribution logic of processing data from multiple clients at the same time. It helped in reducing the execution time by 4 hours.
          \item Helped in migrating the codebase from python 2.7 to python 3.6
      \end{itemize}
      \item \textbf{Stride.ai Inc} | \emph{Summer Intern} \hfill \textit{Jun 2018 - Aug 2018}    
      \begin{itemize}
          \item Developed a data scraping tool using python. It helped in automatically downloading publicly available documents from financial domains and extract key data points.
          \item Tagged data to identify the regions containing tabular data in PDF-based documents. The data helped in training a CNN-based model to extract tabular information. 
      \end{itemize}
  \end{itemize}
}
{
\colorbox{gray}{\makebox[\textwidth-2\fboxsep][l]{\bfseries\color{white} Projects}}
\begin{itemize}
    \item \textbf{Tactile Auditory Learning Kit} \hfill \textit{Jan 2018 - Jul 2018}
    \begin{itemize}
        \item Implemented a CNN-based model to extract edges out of images and developed an algorithm to fit appropriate Braille characters along the edges of the image. It assisted in printing low-cost tactile graphics.
        \item Developed a GUI to assist in converting a PDF-based textbook into its Braille equivalent. 
        \item The project is currently deployed at \textbf{Mathru blind school}, \emph{Bengaluru}.
    \end{itemize}
    \item \textbf{Mathematical Reasoning using MathQA\footnote{\url{ https://math-qa.github.io/}} dataset} \hfill \textit{Aug 2021 - Mar 2022}
     \begin{itemize}
         \item Worked on building a transformer based Fixer module to automatically correct the incorrectly annotated datapoints from MathQA dataset using self-supervision.
         \item The Fixer corrected 16\% of incorrectly annotated datapoints from training set (\href{https://github.com/anushkumarv/Language_models_explore_numeracy}{\textcolor{blue}{link}}) 
     \end{itemize}   
\end{itemize}
}
{
\colorbox{gray}{\makebox[\textwidth-2\fboxsep][l]{\bfseries\color{white} Publication}}
\begin{itemize}
    \item \textbf{Anush Kumar}, Nihal V. Nayak, Aditya Chandra and Mydhili K. Nair. \textbf{Study on Unsupervised Statistical Machine Translation for Backtranslation}. \textit{Proceedings of Recent Advances in Natural Language Processing (RANLP 2019),Varna, Bulgaria}. (\href{https://aclanthology.org/R19-1068/}{\textcolor{blue}{link}}) 
\end{itemize}
}
{
\colorbox{gray}{\makebox[\textwidth-2\fboxsep][l]{\bfseries\color{white} Technical Talk}}
\begin{itemize}
    \item \textbf{Workshop on Software Design Patterns} \hfill \textit{May 2022}
    \\Was responsible for partial delivery of post-graduate (MSWE12) and under-graduate (IS62) \textit{Software Design Patterns} course at \textit{Ramaiah Institute Of Technology}. The workshop covered behavioral and creational patterns.
\end{itemize}
\begin{itemize}
    \item \textbf{Hands on Workshop on using  Git, Github and Node js} \hfill \textit{May 2019}
    \\Conducted a series of workshops to peers, junior year, and sophomore year students as part of the developer student community. It covered version control systems and node js.
\end{itemize}
}
% {
% \colorbox{gray}{\makebox[\textwidth-2\fboxsep][l]{\bfseries\color{white} Awards and Certifications}}
% \begin{itemize}
%     \item \textbf{Smart India Hackathon} \textit{(National level)} \hfill \textit{Mar 2018}
%     \\Inspiration award by \emph{Ministry of Social Justice and Empowerment, Government of India} for implementing \emph{Tactile Auditory Learning Kit} 
%     \item \textbf{ITC info tech codeathon} \hfill \textit{May 2017}
%     \\ Special jury award winners for developing an in-app feedback system using android, express-js, and mongoDB
%     \item \textbf{NPTEL} Appeared among the top 5 percent in the completion of \textbf{Design And Analysis Of Algorithms} in a nationwide conducted test by \emph{Indian Institute of Technology, Madras}\hfill \textit{Mar 2017}
% \end{itemize}
% }
% {
% \colorbox{gray}{\makebox[\textwidth-2\fboxsep][l]{\bfseries\color{white} Skills}}
% }
  
\end{document}
