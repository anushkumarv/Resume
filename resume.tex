\documentclass[11pt,a4paper]{article}
\usepackage[T1]{fontenc}
\usepackage[latin1]{inputenc}
\usepackage{charter}
\usepackage[english]{babel}
\usepackage[centering,includefoot,margin=0.75cm]{geometry}
\usepackage{xcolor}
\usepackage{calc,blindtext}
\usepackage{setspace}
\setlength{\parindent}{0pt}
\usepackage{hyperref}

\begin{document}

\begin{center}
    \textbf{\huge ANUSH KUMAR VENKATESH}
    \\ Email : anushkumarv3 [at] gmail [dot] com
\end{center}

\colorbox{gray}{\makebox[\textwidth-2\fboxsep][l]{\bfseries\color{white} Education}}
    \begin{itemize}
    \item \textbf{Ramaiah Institute Of Technology}\emph{, Bengaluru, India} \hfill \textit{August 2015 - July 2019}
    \\ B.E in Information Science and Engineering \emph{CGPA : 9.48}
\end{itemize}

{
\colorbox{gray}{\makebox[\textwidth-2\fboxsep][l]{\bfseries\color{white} Work Experience}}

  \begin{itemize}
      \item \textbf{J.P.Morgan \& Chase}
      \\ \emph{Associate Software Engineer} \hfill \textit{Jan 2019 - present}  
      \begin{itemize}
          \item[o] Contributed to a new heuristic-based calculator. It takes into account market parameters and determines the risk associated with multiple clients the firm actively trades with.
          \item[o] Refactored the legacy code responsible for the distribution logic of processing data from multiple clients at the same time. It helped in reducing the execution time by 4 hours.
          \item[o] Contributed to migrating the codebase from python 2.7 to python 3.6
      \end{itemize}
      \item \textbf{Stride.ai Inc}
      \\ \emph{Summer Intern} \hfill \textit{Jun 2018 - Aug 2018}    
      \begin{itemize}
          \item[o] Developed a data scraping tool using python. It helped in automatically downloading publicly available documents from financial domains and extract key data points.
          \item[o] Tagged data to identify the regions containing tabular data in PDF-based documents. The data helped in training a CNN-based model to extract tabular information. 
      \end{itemize}
  \end{itemize}
}
{
\colorbox{gray}{\makebox[\textwidth-2\fboxsep][l]{\bfseries\color{white} Technical Talk}}
\begin{itemize}
    \item \textbf{Workshop on Software Design Patterns} \hfill \textit{May 2022}
    \\Was responsible for partial delivery of post-graduate (MSWE12) and under-graduate (IS62) \textit{Software Design Patterns} course at \textit{Ramaiah Institute Of Technology}. The workshop covered behavioral and creational patterns.
\end{itemize}
\begin{itemize}
    \item \textbf{Hands on Workshop on using  Git, Github and Node js} \hfill \textit{May 2019}
    \\Conducted a series of workshops to peers, junior year, and sophomore year students as part of the developer student community that covered version control systems and node js.
\end{itemize}
\begin{itemize}
    \item \textbf{Teaching Assistant for Scripting Languages Lab} \hfill \textit{Aug 2017 - Dec 2017}
    \\Was responsible for conducting weekly beginner level python classes to peers. 
\end{itemize}
}
{
\colorbox{gray}{\makebox[\textwidth-2\fboxsep][l]{\bfseries\color{white} Publication}}
\begin{itemize}
    \item \textbf{Anush Kumar}, Nihal V. Nayak, Aditya Chandra and Mydhili K. Nair. \textbf{Study on Unsupervised Statistical Machine Translation for Backtranslation}. \textit{Proceedings of Recent Advances in Natural Language Processing (RANLP 2019),Varna, Bulgaria}. (\href{https://aclanthology.org/R19-1068/}{\textcolor{blue}{link}}) 
\end{itemize}
}
{
\colorbox{gray}{\makebox[\textwidth-2\fboxsep][l]{\bfseries\color{white} Projects}}
\begin{itemize}
    \item \textbf{Tactile Auditory Learning Kit} \hfill \textit{Jan 2018 - Jul 2018}
    \begin{itemize}
        \item[o] Implemented a CNN-based model to extract edges out of images and developed an algorithm to fit appropriate Braille characters along the edges of the image. It helped in printing low-cost tactile graphics.
        \item[o] Developed a GUI to assist in converting a PDF-based textbook into its Braille equivalent. 
        \item[o] The project is currently deployed at \textbf{Mathru blind school}, \emph{Bengaluru}.
    \end{itemize}
    \item \textbf{Data Science with Julia} - \textit{Jupyter notebooks} \hfill \textit{Aug 2019 - Dec 2019}
    \begin{itemize}
        \item[o] Contributed code snippets for a book explaining data science concepts with \href{https://julialang.org}{\emph{Julia}}. (yet to be published) (\href{https://github.com/anushkumarv/Data_Science_with_Julia}{\textcolor{blue}{link}})
    \end{itemize}
    \item \textbf{Mathematical Reasoning using MathQA\footnote{\url{ https://math-qa.github.io/}} dataset} \hfill \textit{Aug 2021 - Mar 2022}
     \begin{itemize}
         \item[o] Worked on building a transformer based Fixer module to automatically correct the incorrectly annotated datapoints from MathQA dataset using self-supervision.
         \item[o] The Fixer corrected 16\% of incorrectly annotated datapoints from training set (\href{https://github.com/anushkumarv/Language_models_explore_numeracy}{\textcolor{blue}{link}}) 
     \end{itemize}   
\end{itemize}
}
{
\colorbox{gray}{\makebox[\textwidth-2\fboxsep][l]{\bfseries\color{white} Awards and Certifications}}
\begin{itemize}
    \item \textbf{Smart India Hackathon} \textit{(National level)} \hfill \textit{Mar 2018}
    \\Inspiration award by \emph{Ministry of Social Justice and Empowerment, Government of India} for implementing \emph{Tactile Auditory Learning Kit} 
    \item \textbf{ITC info tech codeathon} \hfill \textit{May 2017}
    \\ Special jury award winners for developing an in-app feedback system using android, express-js, and mongoDB
    \item \textbf{NPTEL} Appeared among the top 5 percent in the completion of \textbf{Design And Analysis Of Algorithms} in a nationwide conducted test by \emph{Indian Institute of Technology, Madras}\hfill \textit{Mar 2017}
\end{itemize}
}
  
\end{document}
